%%%%%%%%%%%%%%%%%%%%%%%%%%%%%%%%%%%%%%%%%%%%%%%%%%%%%%%%%%%%%%%%%%%%%%%%%%%%%%%%
%2345678901234567890123456789012345678901234567890123456789012345678901234567890
%        1         2         3         4         5         6         7         8

\documentclass[letterpaper, 10 pt, conference]{ieeeconf}  % Comment this line out if you need a4paper

%\documentclass[a4paper, 10pt, conference]{ieeeconf}      % Use this line for a4 paper

\IEEEoverridecommandlockouts                              % This command is only needed if 
                                                          % you want to use the \thanks command

\overrideIEEEmargins                                      % Needed to meet printer requirements.

% See the \addtolength command later in the file to balance the column lengths
% on the last page of the document

% The following packages can be found on http:\\www.ctan.org
%\usepackage{graphics} % for pdf, bitmapped graphics files
%\usepackage{epsfig} % for postscript graphics files
%\usepackage{mathptmx} % assumes new font selection scheme installed
%\usepackage{times} % assumes new font selection scheme installed
\usepackage{amsmath} % assumes amsmath package installed
\usepackage{amssymb}  % assumes amsmath package installed
\usepackage{color}




\newcommand{\dhm}[1]{\textcolor{blue}{Dylan: #1}}


\title{\LARGE \bf
Preparation of Papers for IEEE Sponsored Conferences \& Symposia*
}


\author{Ankush Gupta$^{1}$, Dylan Hadfield-Menell$^{2}$, Robbie Gleichman, and Pieter Abbeel% <-this % stops a space
\thanks{*This work was not supported by any organization}% <-this % stops a space
\thanks{$^{1}$Albert Author is with Faculty of Electrical Engineering, Mathematics and Computer Science,
        University of Twente, 7500 AE Enschede, The Netherlands
        {\tt\small albert.author@papercept.net}}%
\thanks{$^{2}$Bernard D. Researcheris with the Department of Electrical Engineering, Wright State University,
        Dayton, OH 45435, USA
        {\tt\small b.d.researcher@ieee.org}}%
}


\begin{document}



\maketitle
\thispagestyle{empty}
\pagestyle{empty}


%%%%%%%%%%%%%%%%%%%%%%%%%%%%%%%%%%%%%%%%%%%%%%%%%%%%%%%%%%%%%%%%%%%%%%%%%%%%%%%%
\begin{abstract}
Deformable object manipulation remains a challenging task in robotics.
Continuous, high dimensional, state and action spaces make standard
model-based approaches to manipulation planning intractable. Recent work
has shown progress on this task by learning from
demonstrations through 
\emph{trajectory transfer}~\cite{Schulmanetal_ISRR2013, Schulmanetal_IROS2013}. This approach avoids planning in the state space of a deformable object
by finding a spatial warping that can be used to apply an expert demonstration
to a new scene.

We propose a method for improving trajectory transfer that makes use of bootstrapping examples through simulation. Given a simulator and a way to detect success, we augment our trajectory library with example states and transferred trajectories that have succeeded in simulation. We apply this approach to a simulated overhand knot-tying task. The approach described in Schulman et al.~\cite{Schulmanetal_ISRR2013} achieves a success rate of 59\%. We demonstrate performance of up to 85\%. 

\end{abstract}


%%%%%%%%%%%%%%%%%%%%%%%%%%%%%%%%%%%%%%%%%%%%%%%%%%%%%%%%%%%%%%%%%%%%%%%%%%%%%%%%
\section{INTRODUCTION}
A large challenge in applying standard manipulation and planning techniques to
deformable object manipulation is that of tractable modelling. Deformable object 
are often characterized by high-dimensional, continuous state-action spaces. Model-based 
planning has yet to scale up to the task of efficient planning in this
setting.

Recent work have gained traction on this problem through the technique of 
learning from demonstrations~\cite{Schulmanetal_ISRR2013,Schulmanetal_IROS2013}.
These results are achieved through \emph{trajectory transfer}, where a demonstration
trajectory is generalized to fit a new scenario. Trajectory transfer finds a non-rigid
registration is found an example scene and the current scene and that is
used to modify the demonstration trajectory so that it better fits the scene at hand.
This method of transfer is model-free and obviates the need to plan in complicated
and intractable models of deformable objects. Trajectory transfer has demonstrated 
state-of-the-art performance for knot-tying and suturing.

An important aspect of these strategies is the incorporation of multiple demonstrations
into the process. By increasing the amount of instruction, it becomes possible to do more
tasks. Additionally, demonstrations can take the form of steps in a task and can be
order or recombined to further increase the set of possible successful manipulations.

However, an important problem remains: how should we pick which trajectory to transfer
from a library of options given an input scene? Incorrect selection may lead us
to fail at a task which would otherwise be possible for the correct selection of 
trajectories.

Schulman et al. propose to use nearest-neighbors with registration cost (a measure of the goodness of 
fit for the registration) as a similarity measure to solve this problem. In this paper,
we consider the problem of effectively learning to pick the trajectory to generalize
from experience.

We frame this problem in terms of manifold learning with respect to the state space of our object.
Given a manipulation task, $m$, and demonstration state-trajectory pair, $d, t$, there is region of state space, $S$, 
such that $t$ will perform $m$ successfully when transferred to any $s \in S$. 
In this framework, the nearest-neighbor selection rule represents $S$ with the singlton $d$ and chooses the 
trajectory which minimizes distance to an known example from $S$. \dhm{This is pretty sloppy, and probably too
detailed for the intro but I'm including it so we have it down on paper.} A natural way to extend this approach uses successful traces of trajectory transfer
to draw new states from $S$.

The contributions of this paper are as follows: {\bf(i)} we frame the problem of selecting a trajectory to transfer
from a library as one of estimating distance to a manifold; {\bf(ii)} we propose a method for building a model-free
representation of the manifold associated with a demonstration from successful traces of execution; {\bf(iii)} we show
how this approach can be leveraged to improve finding correspondences with new scenes; {\bf(iv)} we demonstrate
the effectiveness of this approach by showing an improvement of \dhm{Number here} over the nearest-neighbor selection
strategy in a simulated knot-tying task.

\dhm{We need a name}

\dhm{Figures: illustration of segment tree; Graphs/Table showing performance of NN/No Correspondence/Best C-Forest; 
     Graph showing performance of C-Forest as fn of number of training examples; Illustration of manifold idea}





\section{Related Work}
\begin{itemize}
  \item Reinforcement Learning
    \item Deformable Object Manipulation (particularly knot-tying)
      \item Manifold Learning
        \item Anything else?
\end{itemize}


\section{Technical Background}
This is technical background
\subsection{Trajectory Transfer}
\subsection{Reinforcement Learning}
don't know how related this actually is
\subsection{Manifold Learning}


\section{Approach}
In this section, 
\subsection{Transfer Manifolds}
\subsection{Trajectory Selection and a Nearest-Manifold Query}
\subsection{Learning Segment Trees}


\section{Experimental Setup}
\subsection{Experimental Setup}
We evaluate our learning method in a simulation environment on an overhand knot-tying task. We use floating grippers to study the effects of trajectory transfer without the complications of altering derived trajectories to incorporate joint constraints.

\subsubsection{Demonstrations}
The demonstrations we use to initialize the trajectory library are those used by Schulman et al.~\cite{Schulmanetal_ISRR2013} for their experiments. The demonstrations split the task of tying an overhand knot into 3 steps. Demonstrations were collected
by physically guiding a Willow Garage PR2 through these steps and opening or closing the grippers at the appropriate points. There are 36 demonstrations of full knot ties in the data set in addition to several demonstrations that correct for common failures.
Point clouds were collected with an Asus Xtion Pro RGBD camera and filtered by color to extract rope points. 

\begin{figure}
  \includegraphics[width=\linewidth]{figs/cov.png}
  \caption{Example of the steps involved in an overhand knot-tying task in our simulated environment. The standard demonstrations in our trajectory tie a knot as a sequence of 3 steps.}
  \label{fig:knot_steps}
\end{figure}


\subsubsection{Simulation Environment and Task Distribution} 
The tasks we consider are simulations of an overhand knot tying tasks. 
We simulate a rope as a chain of cylinders linked by bending and torsional constraints.
Simulation is done through the use of Bullet Physics engine~\cite{Bullet_Physics}.

Our distribution over initial states is defined procedurally. We begin by uniformly
selecting an initial state from the demonstration. Then 7 points along the rope are 
drawn and subjected to 10cm of perturbation in a random direction. Finally a random
rotation between 0 and $\frac{\pi}{4}$ is applied to the perturbed rope. We ran our
bootstrapping algorithms on initial states from this distribution and tested on a 
separate evaluation set.

\subsubsection{Training and Evaluation}
We generated 10 sets of 170 states each drawn IID from our initial state distribution.
We trained a trajectory library for each of these 10 sets of initial states.
Training was accomplished by an initial exploration phase of 50 attempts where only the initial (expert/ human) demonstrations were used. Then there were 120 knot-tying attempts that chose and transferred trajectories using the new techniques. We evaluated our bootstrapped libraries on an evaluation set that consisted of 300 initial states that were held out from training.

\section{Results}
The section describes the experiments we did and the results we got.


\section{Conclusion and Future Work}
This section summarizes the contributions and suggests some next steps.


\addtolength{\textheight}{-12cm}   % This command serves to balance the column lengths
                                  % on the last page of the document manually. It shortens
                                  % the textheight of the last page by a suitable amount.
                                  % This command does not take effect until the next page
                                  % so it should come on the page before the last. Make
                                  % sure that you do not shorten the textheight too much.

%%%%%%%%%%%%%%%%%%%%%%%%%%%%%%%%%%%%%%%%%%%%%%%%%%%%%%%%%%%%%%%%%%%%%%%%%%%%%%%%



%%%%%%%%%%%%%%%%%%%%%%%%%%%%%%%%%%%%%%%%%%%%%%%%%%%%%%%%%%%%%%%%%%%%%%%%%%%%%%%%



%%%%%%%%%%%%%%%%%%%%%%%%%%%%%%%%%%%%%%%%%%%%%%%%%%%%%%%%%%%%%%%%%%%%%%%%%%%%%%%%


\section*{APPENDIX}

Appendixes should appear before the acknowledgment.

\section*{ACKNOWLEDGMENT}

The preferred spelling of the word �acknowledgment� in America is without an �e� after the �g�. Avoid the stilted expression, �One of us (R. B. G.) thanks . . .�  Instead, try �R. B. G. thanks�. Put sponsor acknowledgments in the unnumbered footnote on the first page.



%%%%%%%%%%%%%%%%%%%%%%%%%%%%%%%%%%%%%%%%%%%%%%%%%%%%%%%%%%%%%%%%%%%%%%%%%%%%%%%%

References are important to the reader; therefore, each citation must be complete and correct. If at all possible, references should be commonly available publications.


\bibliographystyle{plain}
\bibliography{references}


\end{document}
